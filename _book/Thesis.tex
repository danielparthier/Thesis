% Options for packages loaded elsewhere
\PassOptionsToPackage{unicode}{hyperref}
\PassOptionsToPackage{hyphens}{url}
%
\documentclass[
  12pt,
]{book}
\usepackage{lmodern}
\usepackage{setspace}
\usepackage{amssymb,amsmath}
\usepackage{ifxetex,ifluatex}
\ifnum 0\ifxetex 1\fi\ifluatex 1\fi=0 % if pdftex
  \usepackage[T1]{fontenc}
  \usepackage[utf8]{inputenc}
  \usepackage{textcomp} % provide euro and other symbols
\else % if luatex or xetex
  \usepackage{unicode-math}
  \defaultfontfeatures{Scale=MatchLowercase}
  \defaultfontfeatures[\rmfamily]{Ligatures=TeX,Scale=1}
  \setmainfont[]{Nimbus Roman}
  \setsansfont[]{Nimbus Roman}
\fi
% Use upquote if available, for straight quotes in verbatim environments
\IfFileExists{upquote.sty}{\usepackage{upquote}}{}
\IfFileExists{microtype.sty}{% use microtype if available
  \usepackage[]{microtype}
  \UseMicrotypeSet[protrusion]{basicmath} % disable protrusion for tt fonts
}{}
\makeatletter
\@ifundefined{KOMAClassName}{% if non-KOMA class
  \IfFileExists{parskip.sty}{%
    \usepackage{parskip}
  }{% else
    \setlength{\parindent}{0pt}
    \setlength{\parskip}{6pt plus 2pt minus 1pt}}
}{% if KOMA class
  \KOMAoptions{parskip=half}}
\makeatother
\usepackage{xcolor}
\IfFileExists{xurl.sty}{\usepackage{xurl}}{} % add URL line breaks if available
\IfFileExists{bookmark.sty}{\usepackage{bookmark}}{\usepackage{hyperref}}
\hypersetup{
  pdftitle={Medial Septal Projections to the Parasubiculum},
  pdfauthor={Daniel Parthier},
  hidelinks,
  pdfcreator={LaTeX via pandoc}}
\urlstyle{same} % disable monospaced font for URLs
\usepackage[left=3cm, right=3cm, top=2.5cm, bottom=2.5cm]{geometry}
\usepackage{longtable,booktabs}
% Correct order of tables after \paragraph or \subparagraph
\usepackage{etoolbox}
\makeatletter
\patchcmd\longtable{\par}{\if@noskipsec\mbox{}\fi\par}{}{}
\makeatother
% Allow footnotes in longtable head/foot
\IfFileExists{footnotehyper.sty}{\usepackage{footnotehyper}}{\usepackage{footnote}}
\makesavenoteenv{longtable}
\usepackage{graphicx,grffile}
\makeatletter
\def\maxwidth{\ifdim\Gin@nat@width>\linewidth\linewidth\else\Gin@nat@width\fi}
\def\maxheight{\ifdim\Gin@nat@height>\textheight\textheight\else\Gin@nat@height\fi}
\makeatother
% Scale images if necessary, so that they will not overflow the page
% margins by default, and it is still possible to overwrite the defaults
% using explicit options in \includegraphics[width, height, ...]{}
\setkeys{Gin}{width=\maxwidth,height=\maxheight,keepaspectratio}
% Set default figure placement to htbp
\makeatletter
\def\fps@figure{htbp}
\makeatother
\setlength{\emergencystretch}{3em} % prevent overfull lines
\providecommand{\tightlist}{%
  \setlength{\itemsep}{0pt}\setlength{\parskip}{0pt}}
\setcounter{secnumdepth}{5}
\usepackage{booktabs}
\usepackage{float}
\usepackage[format=plain,
labelfont={bf,small,it},
textfont={small,it}]{caption}
\renewcommand\chaptername{}
\pagestyle{plain}
\raggedbottom
\usepackage[]{natbib}
\bibliographystyle{apalike}

\title{Medial Septal Projections to the Parasubiculum}
\usepackage{etoolbox}
\makeatletter
\providecommand{\subtitle}[1]{% add subtitle to \maketitle
  \apptocmd{\@title}{\par {\large #1 \par}}{}{}
}
\makeatother
\subtitle{Inaugural-Dissertation\\
to obtain the academic degree\\
Doctor rerum naturalium (Dr.~rer. nat.)\\
~\\
submitted to the Department of Biology, Chemistry and Pharmacy\\
of Freie Universität Berlin by}
\author{Daniel Parthier}
\date{18.02.2021}

\begin{document}
\maketitle

{
\setcounter{tocdepth}{1}
\tableofcontents
}
\setstretch{1.5}
\hypertarget{prerequisites}{%
\chapter{Prerequisites}\label{prerequisites}}

Placeholder

\hypertarget{intro}{%
\chapter{Introduction}\label{intro}}

Placeholder

\hypertarget{theta}{%
\section{Theta}\label{theta}}

\hypertarget{medial-septum}{%
\section{Medial Septum}\label{medial-septum}}

\hypertarget{parasubiculum}{%
\section{Parasubiculum}\label{parasubiculum}}

\hypertarget{spatial-navigation}{%
\section{Spatial Navigation}\label{spatial-navigation}}

\hypertarget{theta-1}{%
\subsection{Theta}\label{theta-1}}

\hypertarget{materials-and-methods}{%
\chapter{Materials and Methods}\label{materials-and-methods}}

\hypertarget{in-vitro}{%
\section{In-Vitro}\label{in-vitro}}

\hypertarget{slice-preparation}{%
\subsection{Slice Preparation}\label{slice-preparation}}

For slices animals (n = XX PV-Cre mice, n = XX ChAT-Cre mice ) were deeply anaesthetised using isoflurane decapitated, the brain removed and quickly transferred to ice-cold slicing solution. sucrose artificial cerebral spinal fluid (SACF). SACF contained 87 NaCl, 26 NaHCO\textsubscript{3}, 10 Glucose, 50 Sucrose, 2.5 KCl, 1.25 NaH\_2PO\textsubscript{4}, 0.5 CaCl\textsubscript{2}, 3 MgCl\textsubscript{2} · 6H\textsubscript{2}O.
400 µm horizontal slices were produced using a vibratome (VT1200S, Leica Biosystems, Wetzlar, Germany), then transferred and stored in an interface chamber for up to 1-6 h. Slices were perfused with ACSF (119 NaCl, 26 NaHCO\textsubscript{3}, 10 Glucose, 2.5 KCl, 1 NaH\textsubscript{2}PO\textsubscript{4}, 2.5 CaCl\textsubscript{2}, 1.3 MgCl\textsubscript{2} · 6H\textsubscript{2}O) and oxygenated during the whole period.

\hypertarget{slice-recordings}{%
\subsection{Slice Recordings}\label{slice-recordings}}

Cells where identified using XXXX DIC XXXX and recorded using a glass electrode filled with intracellular solution (120 K-Gluconate, 10 Hepes, 10 KCl, 5 EGTA, 2 MgSO\textsubscript{4} · 7H\textsubscript{2}, 3 MgATP, 1 NaGTP, 5 Phosphocreatine Na, 0.2\% Biocytin) to record currents, voltage and later identify cells via Biocytin staining.

all in c/mM

Drugs used:

\hypertarget{results}{%
\chapter{Results}\label{results}}

Here is a review of existing methods.

\hypertarget{discussion}{%
\chapter{Discussion}\label{discussion}}

Here is a review of existing methods.

  \bibliography{book.bib,packages.bib,references.bib}

\end{document}
